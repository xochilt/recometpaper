
%%%%%%%%%%%%%%%%%%%%%%% file template.tex %%%%%%%%%%%%%%%%%%%%%%%%%
%
% This is a general template file for the LaTeX package SVJour3
% for Springer journals.          Springer Heidelberg 2010/09/16
%
% Copy it to a new file with a new name and use it as the basis
% for your article. Delete % signs as needed.
%
% This template includes a few options for different layouts and
% content for various journals. Please consult a previous issue of
% your journal as needed.
%
%%%%%%%%%%%%%%%%%%%%%%%%%%%%%%%
%
% First comes an example EPS file -- just ignore it and
% proceed on the \documentclass line
% your LaTeX will extract the file if required
\begin{filecontents*}{example.eps}
%!PS-Adobe-3.0 EPSF-3.0
%%BoundingBox: 19 19 221 221
%%CreationDate: Mon Sep 29 1997
%%Creator: programmed by hand (JK)
%%EndComments
gsave
newpath
  20 20 moveto
  20 220 lineto
  220 220 lineto
  220 20 lineto
closepath
2 setlinewidth
gsave
  .4 setgray fill
grestore
stroke
grestore
\end{filecontents*}
%
\RequirePackage{fix-cm}
%
%\documentclass{svjour3}                     % onecolumn (standard format)
%\documentclass[smallcondensed]{svjour3}     % onecolumn (ditto)
\documentclass[smallextended]{svjour3}       % onecolumn (second format)
%\documentclass[twocolumn]{svjour3}          % twocolumn
%
\smartqed  % flush right qed marks, e.g. at end of proof
%
\usepackage{graphicx}
\graphicspath{{figures/}}

\usepackage{caption}
\usepackage{subfig}

\usepackage{cite} 
\usepackage[main=english, spanish]{babel}
%\usepackage[utf8]{inputenc}

\usepackage{amssymb, amsmath, amsbsy} % simbolitos
\usepackage{upgreek} % para poner letras griegas sin cursiva
\usepackage{cancel} % para tachar
\usepackage{mathdots} % para el comando \iddots
\usepackage{mathrsfs} % para formato de letra
\usepackage{stackrel} % para el comando \stackbin
%Para cambiar las figuras y tablas en ingles.
\renewcommand{\tablename}{Table} 
\renewcommand{\figurename}{Figure} 
%
% \usepackage{mathptmx}      % use Times fonts if available on your TeX system
%
% insert here the call for the packages your document requires
%\usepackage{latexsym}
% etc.
%
% please place your own definitions here and don't use \def but
% \newcommand{}{}
%
% Insert the name of "your journal" with
%\journalname{Information Sciences Journal}
%

\begin{document}

%\providecommand\foo{}
%\renewcommand\foo{...}


\title{A method for Context-Aware Recommender System}
%\thanks{Grants or other notes
%about the article that should go on the front page should be
%placed here. General acknowledgments should be placed at the end of the article.}

\subtitle{}

%\titlerunning{Short form of title}        % if too long for running head

\author{Xochilt Ramirez-Garcia \and Mario Garcia-Valdez }

\institute{Xochilt Ramirez-Garcia \at
              \email{xochilt.ramirez@gmail.com}           %  \\
%             \emph{Present address:} of F. Author  %  if needed
           \and
           Mario Garcia-Valdez\at
             mariosky@gmail.com
}

%\date{Received: date / Accepted: date}
% The correct dates will be entered by the editor

\maketitle

\begin{abstract} 

In recent years contextual information has been integrated in 
recommender systems in order to get suitable recommendations 
for users, taking in to account factors of found in the enviroment 
and the specific situation where the user is inmerse.
In this paper a context-aware recommender system for restaurants is presented. 
Contextual variables are input to a fuzzy rule based system in a post-filtering
approach. Post-filtering involves two techniques of
recommendation: collaborative and content-based, two common
problems in these algorithms are cold-start and over-specialization.
In order to reduce the effects of these problems 
a hybrid approach is proposed. As a case study the recommender system as well
as the user interface was tested with the goal of improving the user's
satisfaction. The usability test was focused on the hability?? of users to 
handle contextual information.
To evaluate the system's usability an hauristic evaluation
was cunducted, two performance metrics were used: task-success and task-on-time, in order
to measure the level of user satisfaction.  
Also to test the context-aware recommendations
several experiments were cunducted reaching an accuracy of ..
% Falta detallar las pruebas

% Las metricas esas de task-success y task-on-time
% son para medir la satisfaccion del usuario % pero dices que el
%metodo hibrido tendra la ventaja de reduce the weaknesses of
%recommendation techniques % eso como lo evaluamos?
\keywords{Content-Based\and Context \and  Collaborative Filtering\and  
Fuzzy Logic\and  Recommender Systems} 
% \PACS{PACS code1 \and PACS code2 \and
% \subclass{MSC code1 \and MSC code2 \and more} \end{abstract}
\end{abstract}

\section{Introduction} \label{intro} 

Recommender systems are a technique to provide suggestions of 
useful items for a user. The suggestion relates to various
decition-making processes, for instance, what items to buy, what music
to listen to or what on-line news to read. \textit{Item} is the general
term to denote what product or service  the system recomends for each
user. A recommender system normally focus in a single kind of item.
Recommender sistems arise to cover the lack of personal experience or resources to
evaluate the overwhelming number of alternative items that a Web site or an environment 
may offer\cite{resnick1997recommender}.
Recommender systems development initiated from a rather simple
observation: individuals often rely on recommendations provided by
others in making routine and daily decisions\cite{mahmood2009improving},
\cite{mcsherry2009differentially}. 
For example people commonly rely on suggestions when selecting a
book to read; when hiring a human resource in a company, the recruiting decision 
takes in to account recommendation letters; and
when selecting a movie to watch, people tend to read and relies on
movie reviews that a film critic has written.\\
A recommender system tries to imitate this human behaviour, to
recommend a product in daily life is normally the most effective
manner to achieve in this systems that the user prefers an item in a
Web site. 
Over the time, the improvements for recommender systems are
focused in differents techniques that involves the context in the
recommendation process.  The importance of contextual information has
been recognized in different domains and disciplines. In recommender
systems, context implies the situations around of the user when it is
interacting with the system and all the information that this
situation represents.
% definir aqui Context?
The most formal definition is proposed by Annin K.
Dey\cite{dey2001understanding}: \textit{``Context is any information
that can be used to characterize the situation of an entity. An entity
is a person, place or object that is considered relevant to the
interaction between a user and an application, including the user and
applications themselves.''} This definition makes it easier to define
the important context in a specific recommender system.\\ 
However, some authors make an understandable comparison of different 
contexts such as Bazire et.al.\cite{bazire2005understanding} that compares 
different definitions of context in different fields and conclude that is
complicated makes a unifying definition of context because of the
nature of the concept in the disciplines. In computer sciences Fischer
G.\cite{fischer2012context} takes the context such as the interaction
between humans and computers in socio-technical systems that takes
place in a certain context referring to the physical and social
situation in which computational devices and environments are
embedded. Also identifies the important aspects to consider when the
context is used in its framework: how it is obtained the context, how
is represented the context and what objectives and purposes has the
context in a particular application is used. \\
Context-aware recommender systems are gaining ever more attention
because of its performance and adaptability for different domains, the
way to improve personalized recommendations based in contextual
factors is an important technique to increase the benefits in many
domains. Nowadays many companies are incorporating some kind of
context in their recommendation engines, the application covers fields
such as E-commerce\cite{schafer1999recommender}, \cite{bulander2005enabling},
music\cite{ricci2012context},\cite{baltrunas2011incarmusic}, 
places of interest\cite{baltrunas2012context},
movies\cite{eyjolfsdottir2010moviegen}, vacation packages\cite{liu2011personalized},
\cite{liu2014cocktail}, travel guides\cite{savage2012m},e-learning\cite{ortigosa2010entornos} and restaurants\cite{chu2013chinese}.\\  
The majority of these recommender systems are focused on 
recommending relevant items for users without
additional contextual information such as time, location,
companion or place. However, recent systems incorporate the contextual
information to make recommendations in order to deliver items adjusted
to the users' current context. For instance, the site Sourcetone
interactive radio \cite{huq2010automated} when selecting a song for
the customer takes in consideration the listener's current mood in
order to provide a better recommendation. \\ The context is used for
increase important aspects such as user satisfaction, recommendation
quality, usefulness and more.\\
This research presents an architecture proposed for a context-aware
recommender system in the restaurant's domain. The objetive is to
demonstrate how the level of user satisfaction is increased through
the context when the system recommends restaurants in a specific
context. The architecture contains three techniques for this process:
\begin{enumerate}  
\item \textit{Fuzzy Inference System} to recommend such as an expert in
restaurants, it considers the inputs \textit{ratings average},
\textit{price of restaurant} and \textit{number of votes} to infer
how relevant a restaurant is for the user.  
\item \textit{Content-based technique} utilize the restaurant profiles 
to compare how \textit{similar} is a restaurant with respect of another, i.e.  
the restaurants that are \textit{similar} to restaurants 
that the user rated with high rating. The idea is to find restaurants 
with similar features to recommend. 
\item \textit{Collaborative filtering
technique} is based in the user profile to identify the user
preferences and to find neighbors that have the same tastes. The
recommendation consist in the suggestions of other users with similar
tastes that rated restaurants that are not rated for the current user.
A top-N list of restaurants is obtained to recommend for the user. 
\end{enumerate}  
The results of the three techniques are the list of
recommendations for the user that  are adjusted in the user context.
This is the last step and is represented as \textit{context filter} in
the architecture, then the recommender system obtains a list of
contextualized recommendations. The architecture proposed works
simultaneously to obtain recommendations, the hybris method allows the
recommender system to generate suggestions despite the scarse
of user information. When the system face up to the cold-start
problem or the over-specialization problem, the hybrid method
responses properly to show results for the user. An expert 
recommendation contributes to this goal, the rules specified in 
the Fuzzy Inference System helps to inferring the most suitable 
restaurant considering the users' opinions.\\
The context implementation makes better recommendations, the
restaurants matched in the same context of the user will be added
within the final recommendation list. The context-aware recommender
system tries to increase the level of user satisfaction. To meet this
goal two metrics were utilized to test the system in an on-line test.
Ten users were selected to interact with the system, subsequently an
analysis about the system performance and main issues highlited by the
users was realized. \\
The rest of this paper is organized as follows: the state of the art
is explained in next section,  section 3 describes the proposed method
for context-aware recommender system and each one of its components,
the section 4 describes the system evaluarion, the manner how the 
experiments were done and the results obtained, finally, the section 
5 presents the conclusions and future work proposed.
% Recent technologies were utilized to develop a friendly Web interface, 
% for testing the context-aware recommender system. However, 
% the test demonstrates that the design is fundamental
% to achieve a high level of user satisfaction and is not dependant
% of the technolgies used.\\
%Que caracteristicas tiene la arquitectura? 
%Que ventajas tiene?  
%Que y como se va a probar 
%Que hay de novedoso?

%%
%% Esta sección debe llamarse State of the Art, aqui deben ir los otros
%% sistemas de recomendacion sensibles al contexto, de restaurantes, técnicas más utilizadas.
\section{State of the art} \label{sec:2} 

Recommender system is a technology used by many authors for
personalization of information. The amount of information in Internet
is rising quickly and the users every time has less capability to
search the most relevant information among big databases. Many
recommender  system are used in several domains as mentioned in
section 1. The fundamental is the profit that provides in many of
these domains that sometimes represents the key of success for the
application.  Some works utilize social information to recommend such
as M. Manca et.al. \cite{manca2014mining} where the friend recommender
system is applied in the social bookmarking domain, its goal was to
infer the interest of users from content selecting the available
information of the user behavior and analyzing the resources and the
tags bookmarked for each user, therefore the recommendations are
through mining user behavior in a tagging system, analyzing the
bookmarks tagged of the user and the frecuency for each used tag. 
J.Yao et.al. \cite{yao2012product} proposes a new product recommendation
approach for new users based on the implicit relationships between
search keywords and products. The relationships between keywords and
products are represented in a graph and relevance of keywords to
products is derived from attributes of the graph. The relevance
information is utilized to predict preferences of new users. J.
Golbeck et.al.\cite{golbeck2006filmtrust} presents FilmTrust, a
website that integrates Semantic Web-based social networks, augmented
with trust, to create prediction movie recommendations. Trust takes on
the role of a recommender system forming the core of an algorithm to
predict a rating for recommendations of movies. This is an example of
how the Semantic Web, and Semantic trust networks in particular, can
be exploited to refine the user experience. \\  
Traditional recommender techniques has its pros and cons, for
instance, the ability to handle data sparsity and cold-start problems
or considerable ramp-up efforts for knowledge acquisition and
engineering. Establish hybrid systems that combine the strengths of
algorithms and models to overcome some of the shortcomings and
problems has become the properly manner to improve the difficults for
each algorithm.
An example is presented by L.Castro et.al.\cite{castro2012prototype} 
a hybrid recommender system for the province of San Juan, Argentina, 
to recommend tourist packages  based on preferences and interest 
of each user, artificial intelligence
techniques are used to filter and customize the information. The
prototype of recommender system utilizes three techniques to
recommend: demographic, collaborative and content-based. The goal is
to recommend tourist packages that matches with the user profile.
L. Martinez et al.\cite{martinez2009reja} presents REJA, a hybrid
recommender system that involves collaborative filtering and
knowledge-based model, that is able to provide recommendations in some
situation for user; besides it provides georeferenced information
about the recommended restaurants.
Balabanovic et.al. \cite{balabanovic1997fab} presents
Fab, a hybrid recommender system for automatic recognition of
emergent issues relevant to various groups of users. It also enables
two scaling problems, pertaining to the rising number of users and
documents, to be addressed. Claypool et.al.\cite{claypool1999combining} 
presents P-Tango system that utilizes content-based and collaborative
filtering techniques, it makes a prediction through the weighted
average that includes content-based prediction and collaborative
filtering prediction. The weights of predictions are determined on a
per-user basis, allowing the system to determine the optimum mixture
of content-based and collaborative recommendation for each user.
Pazzani M.\cite{pazzani1999framework} presents Entree as a hybrid
recommender system that it does not use numeric scores, but rather
treats the output of each recommender (collaborative, content-based
and demographic) as a set of votes, which are then combined in a
consensus scheme. The recommender system includes information such as
the content of the page, ratings of users and demographic data about
users. Others works with hybrid recommender systems are ProfBuilder
\cite{al1999semantic}, PickAFlick\cite{burke1999integrating}  and
\cite{tran2000hybrid}, where are presented multiple recommendation
techniques. Usually, recommendation requires ranking of
items or selection of a single best recommendation, at this point some
technique must be employed to recommend. \\ 
Traditional recommender systems such as above mentioned, tend to use
simple user models. For example, user-based collaborative filtering
generally models the user as a vector of item ratings. As additional
observations are made about users’ preferences, the user models are
extended, and the user preferences is used to generate
recommendations. This approach, therefore, ignores the notion of any
specific situation, the fact that users interact with the system
within a particular context and  that preferences of items might 
change in another context. \\
Overall, the context is able to make the recommender system be 
powerful that is adaptable to the changing user's situation.\\
The context is defined in the domain of the application and the system
has a context model that provides the information for the recommender
system. For instance Ricci et.al. \cite{baltrunas2011incarmusic} uses
the context in music domain using a model-based paradigm, in this
context-aware recommender system the context was defined as a set of
independent contextual factors(independent in order to get a
mathematical model) such as \textit{driving style, road type,
landscape, sleepiness, traffic conditions, mood weather and natural
phenomena} to specifies the relevant context for the music
recommendation. In order to estimate the relevance of selected
contextual factor, the users were requested to evaluate music tracks
in different contextual situations for each genre. The prediction
takes in account this relevance to recommend music tracks prefered by
the user according the genre and the contexts mentioned.\\ In
restaurant domain Chung-Hua et al.\cite{chu2013chinese} presents a
context-aware recommender system for mobiles using a post-filtering
paradigm, the architecture involves a model client-server that works
with a request of data in the client side for the server side.
Subsequently, taking in account the contextual factors to filter the
properly restaurants to recommend. The context-aware recommender
system uses such as contextual factors \textit{location and season}, 
also utilize the user preferences to personalize the recommendations
in the user context.\\ 
Baltrunas et.al.\cite{baltrunas2011context} presents ReRex for tourism, 
a context-aware recommender system based in a model-based paradigm, the system
recommends and provides explanations about the why the places of
interest(PoI) are recommended. The proposed model computes a
personalized context dependent rating estimation. Subsequently, in
order to generates the explanation of recommendation the system uses
the factor that in the predictive model has the lasgest positive
effect on the rating prediction for the point of interest. The set of
contextual factors considered in ReRex are \textit{distance},
temperature, weather, season, companion, time day, weekday,
crowdedness, familiarity, mood, budget, travel length, transport and
travel goal. The main issue in ReRex system is the low user
satisfaction because of the explanations not able to be understood,
however the users recognize that the explanation is a very important
component that it influence the system acceptance. \\ Noguera et. al.
\cite{noguera2012mobile} presents a context-aware recommender system
for tourism based in REJA that utilizes the location through a 3D-GIS
system, the application uses progressive downloading and rendering of
3D maps over mobiles networks. It is also in charge of tracking the
user’s location and speed based on GPS and the requesting. The system
utilizes pre-filtering and post-filtering paradigm. Pre-filtering is
used to reduce the number of items considered for the recommendation
according to the user’s location, and  post-filtering is applied to
re-rank the previous top-N list according to the physical distance
from the user for each recommended restaurant. The disadvantage in this
system is the lack of user reviews, because the recommendations are
based only in the location point without consider the experience of
other diners concerning the recommended restaurant. 
Cena et al.\cite{cena2006integrating} presents a tourist guide for
context in intelligent content adaptation. UbiquiTO system is a
tourist guide that integrates different forms of context-related
adaptation: \textit{to the media device type, to the user characteristics and
preferences, to the physical context of the interaction}. UbiquiTO uses
a rule-based modeling approach to adapt the content of the provided
recommendation, such as the amount, type of information and features
associated with each recommendation. 
Bulander et.al\cite{bulander2005comparison} presents the MoMa-system that
offers proactive recommendations using a postfiltering approach for
matching order specifications with offers. When creating an order, the
client application will automatically fill in the appropriate physical
context and profile parameters, for example, \textit{location} and \textit{weather},
so that, for example, the facility should not be too far away from the
current location of the user and beer gardens should not be
recommended if it is raining. On the other side, advertisers’
suppliers put offers into the MoMa-system. These offers are also
formulated according to the catalogue. When the system detects a pair
of context matching order and offer, the end user is notified, in the
preferred manner (for example, SMS, email). At this point, the user
can decide whether to contact the advertiser to accept the offer.
Finally, Schifanella et al.\cite{schifanella2008mobhinter} develops
Mob-Hinter, a \textit{context-dependent} distributed model,where a user device
can directly connect to other mobile devices that are in \textit{physical
proximity} through ad-hoc connections, hence relying on a very limited
portion of the users’ community and just on a subset of all available
data (pre-filtering). The relationships between users are modeled with
a similarity graph. MobHinter allows a mobile device to identify the
affinity network neighbors from random ad-hoc communications. The
collected information is then used to incrementally refine locally
calculated predictions, with no need of interacting with a remote
server or accessing the Internet. The Recommendations are computed
using the availables rating of the user neighbors.

%Ejemplos de sistemas en detalle, algoritmos y sus ventajas
%CARS de restaurantes o turisticos

\section{Proposed method}\label{sec:3}
\subsection{Restaurant model} \label{sec:3.1}

An effective online recommender system must be based upon an
understanding of consumer  preferences and successfully mapping
potential products into the consumer’s
preferences\cite{adomavicius2011context}. Pan and
Fesenmaier\cite{pan2006online} argued that this can be achieved
through the understanding of how consumers describe in their own
language a product, a place, and the experience when they are
consuming the product or visiting the place.\\  Traditionally,
choosing a restaurant has been considered as rational behavior where a
number of attributes contribute to the overall usefulness of a
restaurant. For example: food type, service quality, atmosphere of the
restaurant, and availability of information about a restaurant, plays
an important role at different stages in consumer’s desitions
making\cite{auty1992consumer}. While food quality and food type have
been perceived as the most important variables for consumers’
restaurant selection, situational and contextual factors have been
found to be important also. Kivela\cite{jack1997restaurant} identifies
4 distinct types of restaurants: \textit{1) fine dining/gourmet, 2)
theme/atmosphere, 3) family/popular, and 4) convenience/fast-food};
and Auty\cite{auty1992consumer} identifies 4 types of dining out
occasions: \textit{1) namely celebration, 2) social occasion, 3)
convenience/quick meal, and 4) business meal.}
Lewis\cite{lewis1981restaurant}considered five important factors:
\textit{food quality, menu variety, price, atmosphere, and convenience
factors} for a restaurant. Jang et.al.\cite{jang2009perceived}
suggested three factors to consider: \textit{service quality, product
quality and, atmospherics} as restaurant attributes affecting
perceived quality of restaurant experiences. The total dining
experience in a restaurant is comprised of not only food itself, but
also the  \textit{atmosphere, instalations and the service quality}.\\
Taking in account the context proposed by authors, the restaurant
model proposed for context-aware recommender system was unified 55
attributes about the restaurants features. An exploration about
contents of models of others works were compared to define the
suitable information in the model. Therefore, the restaurant model is
a binary vector with the following attributes: \textit{1)price range,
2)payment type, 3)alcohol type, 4)smoking area, 5)dress code,
6)parking type, 7)installations type, 8)atmosphere type, and 9)cuisine
type.} An example of restaurant model in the application is depicted
in Fig.\ref{fig:restaurantmodel} with possible values of the context
represented by a binary vector where 1 means that the restaurant has
the property that corresponds to the position value. Additionally, the
restaurant model contains relevant information such as \textit{reviews
of users, ratings average, and geographycal location}.

\begin{figure*}
\captionsetup{justification=centering,margin=2cm}
\centering
\setlength\fboxsep{0pt}
\setlength\fboxrule{0.7pt}
\fbox{\includegraphics[width=0.75\textwidth]{img/restaurant-model.png}}
\caption{Example of system interface for restaurant model.}
\label{fig:restaurantmodel}   
\end{figure*}
\subsection{User profile} \label{sec:3.2}  

The user's profile is derived from the ratings matrix. Let
$U=[u_1,u_2,...u_n]$ the set of all users and $ I=[i_1,i_2,$...$i_n] $
the set of all items, if $R$ represent the ratings matrix,  an element
$R_{u,i}$ represents a user’s rating $u \in U$  in a item $i \in I$.
The unknown ratings are denoted as $\neq $. The matrix $R$ can be
decomposed into rows vectors, the row vector is denoted as $
\overrightarrow{r_u} $=$[R_{u,1}$...$R_{u,|I|}]$ for every $u \in U$.
Therefore, each row vector represents the ratings of a particular user
over the items. Also denote a set of items rated by a certain user u
is denoted as $ I_u = i \in I | \forall  i: R_{u,i} \neq \emptyset $.
This set of items rated represents the user preferences where for each
domain element $R_{u,i} \in [1-5]$ represents the intensity of the
user interest for  the item.\\  In context-aware recommender system,
user profile has contextual information such as: 1) price range, 2)
current location, 3) cuisine types, 4) attributes or features of
restaurants that the user want, 5) the reviews posted, and 6) the
favorite restaurants list. The user profile is stored in database and
it is available for queries request, and it can be changed by users
many times in a session. The information used to recommendations is
the last one register stored.

\begin{figure*}
\captionsetup{justification=centering,margin=2cm}
\centering
\setlength\fboxsep{0pt}
\setlength\fboxrule{0.7pt}
\fbox{\includegraphics[width=0.75\textwidth]{img/user-profile.png}}
\caption{Example of system interface for user profile.}
\label{fig:user-profile}       
\end{figure*}

\subsection{Expert Recomendation} \label{sec:3.3} 

Fuzzy logic is a methodology that provides a simple way to obtain
conclusions of linguistic data. Is based on the traditional process of
how a person makes decisions based in linguistic information.  Fuzzy
logic is a computational intelligence technique that allows to use
information with a high degree of inaccuracy; this is the difference
with the conventional logic that only uses concrete and accurately
information \cite{zedeh1989knowledge}.\\  In this work, fuzzy logic is
used to model fuzzy variables that highligh in the popularity of a
restaurant. The context-aware recommender system has implemented a FIS
that represents the expert recommendation. The expert FIS generates
recommendations when the recommendation techniques (collaborative
filtering, content-based) are not getting results because of the cold
start problem.\\ The FIS proposed has 3 input
variables\cite{garcia2009hybrid}: 1)\textit{rating} is an average of
ratings that has a particular restaurant inside the user community;
the domain of variable is 0 to 5 and contains 2 membership functions
labeled as \textit{low} and \textit{high}(Fig.\ref{fig:mf:a}),
2)\textit{price} represents what kind of price has a particular
restaurant; the domain of variable is 0 to 5 and contains 2 membership
functions labeled as \textit{low} and \textit{high}
(Fig.\ref{fig:mf:b}), and 3)\textit{votes} is used to measure how many
items have been rated by the current user, i.e., the participation of
the user, if the user has rated few items (less than 10) is not
sufficient to make accurate predictions(Fig.\ref{fig:mf:c}), the
domain of variable is 0 to 10 and contains 2 membership functions
labeled as \textit{insufficient} and \textit{sufficient}. The output
variable is \textit{recommendation}, represents a weight for each
restaurant proposed by the expert considering the inputs mentioned
above, the domain of variable is 0 to 5 and contains 3 membership
functions labeled as \textit{low}, \textit{medium} and \textit{high}
(Fig.\ref{fig:mf:d}).

\begin{figure}[ht!]
   \centering
   %%----primera subfigura----
   \subfloat[]{
        \label{fig:mf:a}
        \includegraphics[width=0.42\textwidth]{img/mf-rating.png}}
   \hspace{0.1\linewidth}
   %%----segunda subfigura----
   \subfloat[]{
        \label{fig:mf:b} 
        \includegraphics[width=0.42\textwidth]{img/mf-price.png}}\\[20pt]
   %%----tercera subfigura----
   \subfloat[]{
        \label{fig:mf:c} 
        \includegraphics[width=0.42\textwidth]{img/mf-votes.png}}
    \hspace{0.1\linewidth}
   %%----cuarta subfigura----
    \subfloat[]{
        \label{fig:mf:d} 
        \includegraphics[width=0.42\textwidth]{img/mf-recommendation.png}}
   \caption{The membership functions of the expert system.
   }
   \label{fig:mfexpert} 
\end{figure}

The proposed FIS in this research(Fig.\ref{fig:expertfis}) represents
the users experience and their knowledge about restaurants. This
factors are considered important  for users that visiting a
restaurant. This information is recovered of user profile and
restaurant profile; and the system uses this information to get
weights that influence in the final recommendations. The FIS uses 5
inference rules that involve the variables of inputs and output.The
input variables determine the recommendation activation; each input
variable contains labels as \textit{low} and \textit{high} that also
correspond to memberships functions of Gaussian type. For the output
variable \textit{recommendation} the labels \textit{low},
\textit{medium}, and \textit{high} are used with membership functions
Gaussian type also. The rules are:

\begin{enumerate} 
\item \textit{If rating is high and price is low then 
recommendation is high.}
\item \textit{If rating is high and votes is sufficient then 
recommendation is high.}
\item \textit{If rating is high and votes is insufficient then 
recommendation is medium.}
\item \textit{If rating is low and price is high and then 
recommendation is low.} 
\item \textit{If rating is low and votes is insufficient then 
recommendation is low.}
\end{enumerate} 

\begin{figure*}
\captionsetup{justification=centering,margin=2cm}
\centering
\fbox{\includegraphics[width=0.75\textwidth]{img/expert.png}}
\caption{Fuzzy Inference System of expert.}
\label{fig:expertfis}     
\end{figure*}

\subsection{Contextual Recommendation} \label{sec:3.4} 

The interface of the system(Fig.\ref{fig:context})  allows to collect
contextual information such as type of price, restaurant's attributes,
type of cuisine and geographical location. The contextual information
is used for adjust the final recommendations list. For example, geographical location is used to get
restaurants around 2 kilometers of distance, next, the list of nearby
restaurants is displayed for the user. If context-aware recommender system
considers another attributes as type of price and type of cuisine preferred by
the user, the system gets restaurants matched in this context, and so on.

\begin{figure*}
\captionsetup{justification=centering,margin=2cm}
\centering
\fbox{\includegraphics[width=0.75\textwidth]{img/context.png}}
\caption{System interface to collect contextual information.}
\label{fig:context}     
\end{figure*}


When the recommendation process is finished, the system displays the restaurants
recommended according the information provides by the user. The context-aware
recommender system contains four techniques to display recommendations. The
interface in Fig.\ref{fig:recom} shows recommendations: 1) Expert, 2)Content-
based, 3) Collaborative filtering and 4)Nearby.

\begin{figure*}
\captionsetup{justification=centering,margin=2cm}
\centering
\fbox{\includegraphics[width=0.75\textwidth]{img/recom.png}}
\caption{System inferface of recommendations section.}
\label{fig:recom}       
\end{figure*}

\subsection{Architecture} \label{sec:3.5}

The architecture for poposed method is depicted in the Fig.\ref{fig:archit}. In
the first part, the three techniques of recommendations are suplied by the
rating matrix to obtain the recommendation list of each one.  In the middle,the
content-based uses cosine similarity to calculate the similarity between the
items, next, collaborative filtering uses the Pearson correlation to calculate
similarity.  In the last part, the recommendation lists for the user.
Subsequently, the recommendation lists are reduced when filter context is
applied, i.e., the recommendations are adjusted for the user context. At the
end, a contextual recommendations list is displayed in the user
interface(Fig.\ref{fig:recom}).

\begin{figure*}
\captionsetup{justification=centering,margin=2cm}
\centering
\fbox{\includegraphics[width=0.75\textwidth]{img/archit.png}}
\caption{Architecture proposed.}
\label{fig:archit}    
\end{figure*}

\section{Evaluation of the system} \label{sec:4}

\subsection{Database}\label{sec:4.1}

The database was collected from Web sites of Tijuana restaurants. There are 176
restaurants for evaluation by 10 real users. The rating matrix contains the
user’s profiles, their tastes and preferences are stored such as a rating’s
vector.

\subsection{Experiments and results}\label{sec:4.2}

To evaluate context-aware recommender system was used the \textit{task success}
and \textit{time-on-task} metrics. \\ The \textit{task success metric} is
perhaps the most widely used performance metric. It measures how effectively
users are able to complete a given set of tasks.  The \textit{time-on-task
metric} is a common performance metric that measures how much time is required
to complete a task\cite{albert2013measuring}.\\ Task success is something that
almost anyone can do.  If the users can’t complete their tasks, then something
is wrong.  When the users fail to complete a simple task can be an evidence that
something needs to be fixed in the recommender system.  The usability test
consist of a list of simple tasks for users that they shall perform in the
system to complete the test. Before to start, a minimal description about the
system for user was explained. The tasks list are the following:

\begin{enumerate} 
\item \textit{Rated a restaurant without context.}
\item \textit{Add context to the user profile.}
\item \textit{Filter restaurants by favorite context.}
\item \textit{Find information of a specific restaurant.}
\item \textit{Find all the reviews of a specific restaurant.} 
\item \textit{Find section of my favorite restaurants.}
\item \textit{Add a review of a restaurant.}
\item \textit{Find the most popular restaurants.}
\item \textit{Add a restaurant to your wishlist.}
\item \textit{Get recommendations based on expert opinion.} 
\item \textit{Get the recommendations content-based.}
\item \textit{Get the collaborative recommendations.}
\item \textit{Get recommendations of the nearby restaurants.}
\end{enumerate} 

The users were students of Tijuana Institute of Technology, they never
interacted with the system interface. To say if the user completed the task was
based in the complete realization of the same without taking in account the time
used. The simple tasks was doing in less than a minute but the result shows that
the user had some problems to complete it.\\    Each user did the test according
the task list, the result is depicted in Fig.\ref{fig:tsuccess} where the axis
represent the task number and percent of success. The chart shows that only 3
tasks weren't accomplished successfuly, the task 5, 6 and 7. The issue with task
5 and task 7 was that users can not found easily the reviews and add it was
complicated also. In task 6 the issue was that the user chose the wishlist
section in place of the favorites restaurants section.\\

\begin{figure*}
\captionsetup{justification=centering,margin=1cm}
\centering
\fbox{\includegraphics[scale=0.75]{img/tsuccess.png}} %[width=0.7\textwidth]
\caption{Representation of the percent of success for each task.}
\label{fig:tsuccess}   
\end{figure*}

The time it takes a participant to perform a task says a lot about the usability
of the application. In almost every situation, the faster a participant can
complete a task, the better the experience. In fact, it would be pretty unusual
for a user to complain that a task took less time than expected
\cite{albert2013measuring}.\\ Then, task-on-time was applied to measure time
that an user did the task. A resume of the time tasks for each user it is in
Table \ref{tab:datausers}, null values mean that the user didn't the task.


\begin{table}
\centering
\caption{Time on task data for 10 users and 13 tasks. }
\label{tab:datausers}  
\begin{tabular}{lllllllllll}
\hline\noalign{\smallskip}
Task  & Us1  & Us2 & Us3 & Us4 & Us5 & Us6 & Us7 & Us8 & Us9 & Us10 \\
\noalign{\smallskip}\hline\noalign{\smallskip}
1 & 12  & 28 & 24 & 30 & 19 & 33  & 23 & 16 & 5  & 7 \\
2 & 3   & 4  & 17 & 5  & 17 & 134 & 9  & 16 & 12 & 11 \\
3 & 123 & 69 & 159& 53 & 69 & 113 & 44 & 41 & 70 & 98 \\
4 & 20  & 4  & 86 & 40 & 13 & 4   & 17 & 3  & 20 & 3 \\
5 & 50  & 10 & 63 & 50 & 7  & 11  & 10 & 5  & 20 & Null \\
6 & 10  & 30 & 28 & 27 & 5  & 46  & Null  & 7  & Null  & 34 \\
7 & 10  & 20 & 16 & 8  & 15 & Null   & 9  & 24 & 16 & 28 \\
8 & 18  & 24 & 10 & 10 & 5  & 3   & 27 & 4  & 5  & 6 \\
9 & 5   & 6  & 31 & 4  & 45 & 9   & 12 & 5  & 3  & 8 \\
10 & 15 & 17 & 15 & 11 & 10 & 19  & 13 & 10 & 20 & 20 \\
11 & 30 & 15 & 20 & 16 & 20 & 22  & 15 & 13 & 18 & 20 \\
12 & 12 & 14 & 19 & 14 & 40 & 10  & 17 & 17 & 15 & 15 \\
13 & 25 & 15 & 15 & 14 & 10 & 10  & 11 & 10 & 10 & 25 \\

\noalign{\smallskip}\hline
\end{tabular}
\end{table}

To measure the efficiency of the metric it was chose an confidence interval.  In
this way, it is observed the time variability within the same task and  also
helps visualize the difference across tasks to determine whether there is a
statistically significant difference between tasks. The obtained information is
in Table \ref{tab:ic}, the median was used to calculate the confidence interval.

\begin{table}
\centering
\caption{Table of confidence interval per task with a confidence level of 95\%. }
\label{tab:ic}    
\begin{tabular}{lllll}
\hline\noalign{\smallskip}
Task  & Median & CI 95\% & Upper bound & Lower bound  \\
\noalign{\smallskip}\hline\noalign{\smallskip}
1 &    20         & 5.96  & 25.96 & 14.04  \\
2 &    11.5      & 0.81  & 12.31  & 10.69   \\
3 &    69.5      &  25.57   &  95.07  &  43.93   \\
4 &    15        & 16.34  &  31.34  &  -1.34   \\
5 &    15.5     &  14.84  &  30.34  &  0.66  \\
6 &     27.5    &   11.57  &  39.07  &  15.93    \\
7 &     16       &  5.19  & 21.19  &  10.81   \\
8 &     8         &   5.80  &  13.80  & 2.20 \\
9 &     7         & 9.43  &  16.43  &  -2.43  \\
10 &   15       &   2.44  &  17.44   &  12.56   \\
11 &   19       &  3.00  &  22.00  &  16.00   \\
12 &   14.5    &  5.51  &  20.01  &  8.99   \\
13 &   12.5    &  3.89  &  16.39  &  8.61    \\

\noalign{\smallskip}\hline
\end{tabular}
\end{table}

In the next step the USE \textit{(Usefulnes, Satisfaction, and Ease of Use)}
questionnaire \cite{morris2001experience} was applied in order to get the user's
feedback and comments for to know about the difficults in the test.  The USE
questionnaire consists of 30 rating scales divided into 4 categories:
Usefulness, Satisfaction, Ease of Use, and Ease of Learning. Each is a positive
statement to which the user rates level of agreement on a 7-point Likert scale.
The USE questionnaire allows to get values for Usefulness, Satisfaction, Ease of
Use, and Ease of Learning, the visualizing the results is in the
Fig.\ref{fig:radial} , where the four axis of the radar chart represent the
values of percent which users rated positively this factors with respect to
their interaction with the context-aware recommender system.

\begin{figure*}
\captionsetup{justification=centering,margin=1cm}
\centering
\includegraphics[scale=0.5]{img/radial.png} %[width=0.7\textwidth]
\caption{The radar chart that depicts the four axis evaluated in the questionnaire.}
\label{fig:radial}   
\end{figure*}

\section{Conclusions and future work} \label{sec:5}

We observed the users behaviour to identify the most frecuently difficults and
doubts about tasks. We did a brief interview with users after the test in order
to understand their  feelings or mood, their ideas about the experience, and
overall, their opinion about the context-aware recommender system.  The
conclusions are based in user's comments, then the main errors in the system
interface are summarized in three points:
\begin{enumerate}  
\item  Incomplete information for user, i.e., the system doesn't had enough 
and clear information to be a friendly interface, and therefore the user couldn't 
do easily a task.
\item Fails in design, because of unordered elements in the screen, in other words, 
the elements are not in the correct site into the screen to be easily identified per users.
\item Fails in the language and confusion, because of the english language is not 
the native language of the users.
\end{enumerate}

The three points mentioned are related to the null values in data table (see
Table \ref{tab:datausers}), some users didn't the task because they were
confused, so they decided to omit the task. The null values weren't took in
account when the median was calculated (see Table \ref{tab:ic}).\\  The USE
questionnaire was useful to identify the weaknesses in the context-aware
recommender system.  The percent is upper of the acceptable (80\%), the results
allow to say that the system has a good performance. \\ For the future work we
proposed to improve the problems found in the user interface, so the proposals
are the following:

\begin{enumerate}  
\item  Redesign the user interface could helps to be more friendly for users. 
Due to the issues, the redesign involves: 
  \begin{enumerate}  
  \item Analyze the amount of information enough for a easy understanding, i.e., 
  how much information the user needs seeing without overload it.
  \item Modify the tasks descriptions in the most simple way to avoid confusion.
  \item Add more language functionalities for to facilitate the tasks for users.
  \end{enumerate}
\item  To apply the usability test again with the changes in the interface in 
order to observe the level of improves and to compare the results. 
\item  Apply an statistical test to analize the results.
\item  Add collaborative filtering based on model (matrix factorization technique) 
within the context-aware recommender system in order to improve the level of user 
satisfaction in the context. 
\item  Add any contextual factors (such as companion, time of day, budget, etc.) 
in order to include more context information that could be relevant in the recommendations.
\end{enumerate}

%\begin{acknowledgements}
%If you'd like to thank anyone, place your comments here
%and remove the percent signs.
%\end{acknowledgements}

% BibTeX users please use one of
%\bibliographystyle{spbasic}      % basic style, author-year citations
%\bibliographystyle{spmpsci}      % mathematics and physical sciences
%\bibliographystyle{spphys}       % APS-like style for physics
%\bibliography{}   % name your BibTeX data base

\bibliographystyle{plain}

\bibliography{biblio}

\end{document}